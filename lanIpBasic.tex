\documentclass{article}

\newcommand{\lip}{lanIpBasic}
\newcommand{\ethn}{ethernet}
\newcommand{\Ethn}{Ethernet}
\newcommand{\pad}{{\em PAD}}
\newcommand{\rbuf}{{rbuf}}
\newcommand{\Rbuf}{{Rbuf}}
\newcommand{\lipc}[1]{{\tt #1}}
\newcommand{\lipf}[1]{{\tt #1}}
\newcommand{\cmd}[1]{{\tt #1}}

\begin{document}
\section{Introduction}
\lip{} is a simple implementation of the UDP protocol -- with some
limitations -- designed to run on an \ethn-LAN. For convenience,
a ``pseudo-IP'' layer has been implemented, too, so that the \lip{}
protocol stack can coexist and communicate with ordinary TCP/IP on
a LAN. No real IP features such as routing, fragmentation etc. are
provided, however.

\lip{} allows several nodes on a \ethn-LAN to communicate without much
overhead and in a pseudo-deterministic%
\footnote{
\Ethn{} is not ideally suited for applications which need strict
determinism due to features such as collisions nor is it very efficient
for transmitting very small packet sizes. Some of these drawbacks are
mitigated in a switched, full-duplex environment.
} fashion. \lip{} runs independently from ordinary TCP/IP; there exists
absolutely no coupling between the two.

The API of \lip, however, was designed to stay close to the BSD socket
API for sake of convenience.

  \subsection{History}
\label{sec:history}
\lip{} was originally developed for a very special, small-scope
application and some of the design decisions that were made must
be understood given \lip{}'s history.

When the LCLS BPM project -- due to a tight schedule and unavailability
of a suitable commercial product -- at the time decided to use the 
same digitizer (AKA, ``\pad'') that had been developed for the LCLS RF project
it became evident that a dedicated communication path between the \pad{}
and it's associated main processor was needed. It was decided to add
a second ethernet controller to the \pad{} board which was reserved
for this task.

The design of the communication software laid the foundation for \lip{}
as it was decided to add minimal IPV4 and UDP protocol headers rather than
communicating over raw ethernet. This allowed for development of the
higher application layers as well as \lip{} itself in an ordinary
TCP/IP environment where many diagnostics and development tools are
available thus permitting rapid deployment of \lip{} and the main
applications under a tight schedule.

Only later, other projects, most notably the LCLS fast-feedback project
decided to use \lip{} as a platform and its functionality was subsequently
expanded (e.g., by adding multicast- and IGMP support).

\section{lanIpBasic}
  \subsection{Design Goals}
The main purpose of \lip{} is providing a fast means of
communication over \ethn{} with quasi-deterministic
timing. Due to the complexity of the ``ordinary'' TCP/IP
networking stack and the intrinsic coupling of the many
pieces and users of this stack it is almost impossible
to assert the required deterministic behavior. Also, the
high complexity might not yield the required fast response
time on low-end CPUs (and such a CPU was present in the 
system for which \lip{} was originally developed).

Therefore, \lip{} was developed as ``standalone'' software
which is completely independent from TCP/IP although it
borrows some abstract concepts (such as e.g., sockets) and
implements or supports some of the well-known protocols.

However, rather than just developing and implementing a
``raw ethernet'' protocol we felt that it would be useful
to add minimal IPV4 and UDP functionality. This allows
for \lip{} to interact with ``normal'' UDP sockets which
was very useful for development and testing. Also,
high-level applications can be written so that they
are agnostic to the existence of \lip{} which then
can be used as a ``plug-in replacement'' for TCP/IP.

The decision to incorporate basic IPV4 functionality into
the design proved to be a good one when the request for
supporting multicasting arose. Flooding of multicast
packets on a switched \ethn{} LAN can be reduced by a
technique called ``IGMP snooping'' which is implemented
on commercial switches. However, IGMP snooping requires
the receivers of multicast traffic -- in this case \lip{} --
to participate in the IGMP protocol which is part of
the IP protocol suite. It would have been much harder
to add IGMP support to \lip{} if there hadn't been basic
IP support from the beginning.

The most important design goals can be summarized as
follows:
\begin{itemize}
\item Standalone software, independent from ordinary TCP/IP software stack.
\item Simple, rather than a general-purpose solution. This requirement
      has some implications:
      \begin{itemize}
      \item Should execute efficiently on low-end CPU.
      \item Easier and faster to develop with less risk for bugs.
      \item Some features are hardcoded and more difficult to change.
      \end{itemize}
\item Support basic IPV4 functionality. Full standard compliance is
      {\em not} required; just basic features so that cooperation
      with other nodes on a LAN is possible. Some features that
      can be missing include
      \begin{itemize}
      \item Fragmentation.
      \item Routing.
      \item IP options.
      \item Full ICMP support.
      \end{itemize}
\item Support IGMP ECHO REQUEST so that a \lip{} node can be ``ping''ed.
\item Full support for IGMP V2.
\item Basic UDP support. However, support for packets larger than the
      \ethn{} MTU is not required. Neither is support for the UDP checksum
      required.
\item ARP support. Even though ARP lookups violate the requirement of
      deterministic behavior support for ARP was considered useful - especially
      for testing purposes or applications with less strict requirements.
      Tight control over how ARP is done should be possible (explicit lookup,
      prohibit implicit lookups, user-control over cache-aging).
\item ``Zero copy'', i.e., RX and TX buffers can be passed to/from the
      user directly to/from the hardware/driver w/o the need for intermediate
      copying.
\item Diagnostics and debugging features such as statistics counters, and
      printed messages that can be enabled in ``debug-mode''.
\end{itemize}

  \subsection{Programming Interface}
The latest version of the API documentation is always present in
the \lipf{<lanIpBasic.h>} header file.

IPV4 addresses and port numbers are passed to several routines.
It should be noted that IPV4 addresses
are sometimes passed as character strings (in IPV4 ``dot-notation'')
but mostly as 32-bit unsigned integers (\lipc{uint32\_t}). When encoded
as such integers the representation is always in {\em network byte order}
(AKA ``big-endian''). UDP port numbers, however, are always passed in
{\em native (CPU) byte order} and are internally converted when necessary.
Link-level (``ethernet'') addresses are passed as arrays of
six binary octets ({\em not} as character strings).

  \subsection{Configuration and Initialization}
Before \lip{} can be used it must be properly set up and initialized.
This procedure consists of a few steps:
\begin{enumerate}
\item Configure basic \lip{} parameters with \lipc{lanIpBscConfig()}
  \begin{itemize}
    \item TX and RX descriptor ring size. This information is passed
          down to the driver which may (or in the case of very simple
          hardware may not) use it.
    \item Number of \rbuf{}s that is allocated initially. More buffers
          can always be added at run-time (\lipc{lanIpBscAddBufs()}).
    \item Depth of packet queue per UDP socket. This is the number
          of received UDP datagrams that \lip{} may store in a socket
          before the user picks them up (\lipc{udpSockRecv()}).
  \end{itemize}

  Parameters can only be set {\em prior} to initializing \lip{}.
  However, the routine can be called at any time in order to
  read-back the current configuration.

  Use of this call is optional; \lip{} picks default values for
  all configurable parameters.
\item Initialize \lip{} proper by executing \lipc{lanIpBscInit()}.
  This routine doesn't take any parameters but the return value
  should be checked (non-zero indicates failure due to inability
  to allocate vital resources). If this step fails then \lip{} cannot
  be used.
\item Create a driver instance. Because currently there can only be
  a single driver which is hard-compiled into \lip{} (see
  section~\ref{sec:history} about history) there is no run-time
  choice to make regarding what driver is to be used.

  The driver is instantiated by \lipc{lanIpBscDrvCreate()}. This
  routine is passed a {\em device-}instance and an optional link-level
  (``ethernet'') address. The (zero-based) device instance identifies 
  a specific device in case multiple hardware devices supported by
  the driver are present. If the instance $-1$ is requested then the
  first available instance is used (e.g., a device already attached
  to the regular TCP/IP stack of the operating system would be
  regarded ``unavaliable'' and skipped).
\item Create an ``interface'' instance by executing \lipc{lanIpBscIfCreate()}.
  A handle to the driver instance created in the previous step is passed
  as well as the IPV4 address and netmask to be used by the interface.
  Both of these are passed as strings in IP ``dot'' notation.
\end{enumerate}

A convenience wrapper routine \lipc{lanIpSetup()}
(see \lipc{<lanIpBasicSetup.h>}) is available which performs all of
the above steps (except for \lipc{lanIpBscConfig()}). Unless
there are special user requirements (e.g., using a specific device
instance) the convenience routine should be sufficient for
most applications.
  \subsection{Limitations}
%TODO
  \subsection{Implementation}
This subsection describes some basic objects inside 
\lip{} and other implementation details.
    \subsubsection{\Rbuf{}s}
All payload data and protocol headers in \lip{} are stored
in simple memory buffers called {\em \rbuf{}s}. \Rbuf{}s are
properly aligned, contiguous blocks of memory of a fixed size,
slightly larger than the \ethn{} MTU. Since \lip{} is intended for 
special-purpose applications (i.e., the total number of \rbuf{}s
in use at a time is small to moderate) on modern CPUs with several
if not many MBs of memory it seems reasonable to sacrifice 
some memory-efficiency for sake of simplicity.%
\footnote{
  However, if \lip{} was to be redesigned from scratch we would
  consider a slightly different approach and manage headers and
  payload separately. This would allow for efficient re-use of
  headers with a changing payload. Most modern \ethn{} hardware
  supports scatter/gather DMA or similar techniques so that
  non-contiguous buffers could be handled seamlessly.%
}

While unused, \rbuf{}s are maintained on a linked, ``free-'' list
from where they are allocated either by the user or \lip{} 
internally.

Several API entry points are defined for filling protocol
headers into an \rbuf, accessing the payload and performing
other operations.

New \rbuf{}s can be added to the pool at run-time should there
be a shortage.
	\subsubsection{Interface}
\lip{} implements a ``network interface'' object which is an abstraction
of the underlying \ethn{} hardware. Even though there are provisions
for adding support for multiple interfaces to \lip{}, currently there
can only be one single interface and there are still a few places
in \lip{} relying on the assumption that this restriction applies.

The interface object contains the following parts
\begin{itemize}
\item Pointer to the driver software which handles the hardware. However,
      due to historical reasons and still in the current version of \lip{}
      the driver entry points are actually {\em hard-compiled} into \lip{}
      (by means of macros) and the interface's driver pointer merely points\
      to a driver-internal data structure. 

      In the future, it would be easy to completely upgrade this design
      and make the driver and generic \lip{} code strictly separate modules.
\item IPV4 address.
\item \ethn{} address.
\item IPV4 multicast group addresses the interface is a member of.
\item ARP cache.
\item Statistics counters.
\item Mutex for synchronizing access to interface data.
\end{itemize}

    \subsubsection{Sockets}
A \lip{} socket is similar in concept to a BSD socket, i.e., it is
an abstraction for a communication endpoint. Also, like under BSD,
the user deals with ``socket descriptors'' which are small integer
numbers (internally used as an index into a table of socket objects).
However, \lip{} and BSD sockets exist in a separate, completely
disjunct ``space'' with no functional overlap. E.g., BSD socket
descriptor $2$ refers to something entirely different from \lip{}
descriptor $2$.

Implicitly, \lip{} sockets are always ``datagram'' (UDP) sockets
and the IPV4 ``domain''. A \lip{} socket contains the following
data:
\begin{itemize}
\item Pointer to the interface this socket is using.
\item UDP port number.
\item Message queue holding received packets that have not yet been
      fetched/received by the user. When the hardware receives a new
      packet then the driver sends it up the \lip{} stack which parses
      the headers and when detecting a UDP message being sent to this
      socket's port posts it to the end of this message queue. If the
      queue is full (i.e., if the queue already holds the pre-configured
      maximum of messages) then the datagram is discarded.

      When the user ``receives'' a message from the socket then
      \lip{} removes the first packet from the queue and hands
      it over to the user.
\item Peer address (for a ``connected'' socket).
\item Lock for serializing access to the socket. Most but not all
      operations on a socket are thread-safe (consult the API section
      for details). The (rather rare) exceptions
      (which exist for sake of increased efficiency) are:
      \begin{itemize}
        \item Deleting a socket that is in use (a second thread is
              e.g., sending or receiving while a first thread deletes
              a socket).
        \item Querying the number of bytes available on a socket
              (\lipc{udpSockNRead()})
              from one thread while a second thread is reading from
              the same socket.
      \end{itemize}
\item Flags.
\end{itemize}
Note that a \lip{} socket is {\em always} ``bound'' (in the BSD sense)
 to an address since it always references an interface and holds a port number.

    \subsubsection{Drivers}
\lip{} defines an interface to the driver software
which interacts with specific \ethn{} hardware.
For historical (and efficiency) reasons the driver entry points are
a set of {\em macros} or {\em global subroutines}
which must be defined by the
driver and which are compiled into \lip{} proper.%
\footnote{
Again, if \lip{} was to be redesigned from scratch
this software interface would be an area that would be designed
differently.}

The names of driver entry points all start with the
string \cmd{NETDRV\_} and the driver source file has
the following structure (note that it must include
the file \lipf{lanIpBasic.c} which contains most of
the \lip{} code):

\begin{verbatim}
/*
 * Inclusion of vital headers (only) that are needed for forward
 * declarations of driver entry points:
 *
 * Do not include any other headers here!
 */
#include <stdint.h>

/*
 * Forward declaration of interface struct which is defined
 * by lanIpBasic.c so that it can be used by forward declarations
 * of driver entry points.
 */
struct IpBscIfRec_;

/* Definition of driver entry points which are macros, e.g:            */
#define NETDRV_READ_INCREMENTAL(pif, ptr, nbytes)  do {} while ()

/* Forward declarations of driver entry points which are subroutines:  */
static void
NETDRV_READ_ENADDR(struct IpBscIfRec_ *pif, uint8_t *buf);

/* Definition of driver-specific API header                            */
#define NETDRV_INCLUDE "drvxxx.h"

/* Inclusion of lanIpBasic bulk code */
#include "lanIpBasic.c"

/* Definition of driver entry points which are routines and other code */

static void
NETDRV_READ_ENADDR(struct IpBscIfRec_ *pif, uint8_t *buf)
{
	/* Implementation goes here */
}

\end{verbatim}

The choice whether a given entry point is best implemented as
a macro or subroutine is up to the driver developer.

    \subsubsection{Tasks and Locks}
This subsection describes the fundamental tasks which
execute the \lip{} stack and locks that are in place to
synchronize access to critial data structures. Note
that the driver creates additional tasks and locks which
are driver-specific and not described here in detail for
all available drivers.

\lip{} creates two tasks the ``callout'' task which is used
to schedule future work and the ``work'' task which is used
to schedule low-priority work that may be deferred.

The callout task runs at RTEMS priority $25$ which is relatively
high. A ``callout'' is an object holding timing information, a 
pointer to a user callback routine and user-provided arguments
to that callback routine. The user may schedule a callout which
means that the callback routine will be executed by the callout
task when the predefined timeout expires. Callouts can be
cancelled or rescheduled etc.

\lip{} uses callouts to implement the various timers involved
with the IGMP protocol.

The work task runs at the relatively low RTEMS priority $180$
and deals with low-priority aspects of the IP protocol suite
such as ARP, ICMP and IGMP so that UDP receives the highest
attention. When the main protocol dispatcher \cmd{lanIpProcessBuffer()}
(of the receive path) discovers that the \rbuf{} it is working on 
contains an ARP, ICMP or IGMP packet then it queues the \rbuf{}
onto the list of work for the ``work'' task which will deal
with the packet when no high-priority task needs to execute.

Most drivers create a RX (receiver) task which executes
\lip{}'s main protocol dispatcher (\cmd{lanIpProcessBuffer()}) when
a new packet arrives. A driver also might use a mutex to
protect it's internal data structures in order to make
the driver enty points thread-safe. This is a requirement
since \lip{} may call the entry points from different
task contexts.

Some drivers create additional tasks, e.g., to cleanup
TX DMA rings when transmission has finished or to handle
state changes of the physical link. Other drivers may
handle such events from the same task context that deals
with reception.

\lip{} internally uses four mutexes:
\begin{description}
\item[ARP Lock] serializing access to the ARP cache of an interface.
The ARP lock is acquired when an ARP entry is created or deleted but
also when an entry is looked-up (regardless whether the lookup
hits or misses in the cache).
\item[MC Lock] serializing access to the list of multicast addresses
of an interface. This lock is acquired by the IGMP protocol state machine
but also when the user joins or leaves a multicast group.%
\footnote{Currently, the ARP and Multicast locks are actually the {\em same}
  lock as we felt that an extra mutex could be avoided - however, the code
  is written in a way that would make it trivial to introduce a separate
  mutex.}
\item[Callout Lock] is held by the callout task while walking the list
of callouts belonging to a particular time-slot. This lock protects
the list when the user concurrently tries to cancel a callback.
Note that code which needs to cancel callouts must be written
carefully in order to avoid deadlocks. Make sure to read the comments
in \lipf{lanIpBasic.c} where the \lipc{COLOCK} macro is defined.
\item[Socket Lock] serializing access to a particular socket. Every
open socket has an individual mutex associated. All socket operations
but \lipc{udpSockDestroy()} acquire this mutex to ensure thread-safety.
It is a programming error to destroy a socket that is still being used.
\end{description}

    \subsubsection{Data Flow}
In this subsection we describe the basic flow of packet data
through \lip{}.

\Ethn{} frames are transmitted for the following reasons:
\begin{itemize}
\item The user explicitly sends a UDP datagram. Depending on the call chosen
      by the user \lip{} allocates a fresh \rbuf{} or is passed one that
      already contains the UDP payload by the user. \lip{} fills in \ethn{},
      IPV4 and UDP protocol headers, copies the UDP payload (unless the
      user already passed a \rbuf{}), computes the IP checksum and hands
      the \rbuf{} down to the driver for transmission. Note that \lip{}
      needs to look up the \ethn{} address of the destination in the ARP
      cache and initiate a ARP lookup if the desired address is not found
      in the cache.

      If an ARP lookup must be done and the socket is {\em not connected}
      then the socket mutex is relinquished while the sender waits for
      the ARP reply so that another thread could still send (or receive)
      from the same socket.

      In the case of a {\em connected} socket (which can only send to
      or receive from the peer) the socket mutex is not given up while
      waiting for an ARP reply because no other thread could send until
      the ARP reply from the peer arrives anyways. If automatic updating
      of the ARP cache is enabled then reception of any frame from the
      peer has the same effect as reception of an ARP reply so that 
      a sender holding the socket mutex across an ARP lookup doesn't
      unnecessarily delay a receiving thread either.
      
\item \lip{} broadcasts an ARP REQUEST. This is necessary if a lookup
      in the ARP cache misses and an ARP lookup becomes necessary. 
\item \lip{} sends an ARP REPLY as a response to receiving an ARP REQUEST
      which matches the receiving interface's IPV4 address.
\item \lip{} sends IGMPV2 messages.
\item \lip{} sends an ICMP ECHO REPLY message as a response to receiving
      an ICMP ECHO REQUEST. This feature is intended for testing/debugging
      so that \lip{} can be ``pinged''.
\end{itemize}

Packet reception is usually handled in the context of a driver task
which executes \cmd{lanIpProcessBuffer()} on a received frame.
This routine handles ARP, ICMP (echo requests only), IGMP and queues
UDP packets on a socket's message queue if the port number of a socket
matches the UDP destination port of the packet. Any thread blocking
for data to arrive (in \cmd{udpSockRecv()}) then becomes ready to run
and eventually dequeues the packet from the socket's queue.

As already mentioned ARP, ICMP and IGMP are not processed directly
by \cmd{lanIpProcessBuffer()} but are put on the work queue to
be dealt with by the ``low-priority work task'' thus ensuring
that critical UDP packets reach their destination as fast as 
possible.

    \subsubsection{ARP}
\lip{} maintains an ARP cache where associations between \ethn{} and
IPV4 addresses are stored. Cache entries can be created automatically
or explicitly by the user. Implicit creation of entries happens under
the following circumstances
\begin{itemize}
\item The destination address passed to \lipc{udpSockConnect()},
      \lipc{udpSockHdrsInit()}, \lipc{udpSockHdrsInitFromIf()},
      \lipc{udpSockSend()}, \lipc{udpSockSendBuf()}, \lipc{udpSockSendTo()}
      or \lipc{udpSockSendBufTo()} misses in the
      cache and the implicit ARP lookup that results is successful.
\item If the global variable \lipc{lanIpBscAutoRefreshARP} is nonzero (default)
      then arrival of a UDP or ICMP ECHO REQUEST packet results in a 
      cache entry for the sender of the frame being created or refreshed.
\end{itemize}
Note that by default ARP cache entries {\em do not expire}. This has
the implication that e.g., \lipc{udpSockSend()} never has to perform
an ARP lookup since a successful \lipc{udpSockConnect()} that must 
preceed \lipc{udpSockSend()} already created a cache entry.

However, \lip{} provides a routine \lipc{arpScavenger()} which implements
an algorithm for proper cache ageing. It is up to the user to create a
task and let it execute the \lipc{arpScavenger()}. Note that while not
letting the cache age minimizes ARP lookups this policy carries the risk that
changes of networking hardware and/or IPV4 addresses etc. are not
dealt with transparently but cause communication failure. If the
\lipc{lanIpBscAutoRefreshARP} feature is enabled then ARP lookups
may be minimized even if cache ageing is started provided that 
packets arrive from the relevant peers at a sufficient rate.

Note also that the automatic refreshing of the ARP cache is
handled in the context of the ``work task'', i.e., a cache entry
might not be present yet when a user task receives a datagram
(because the low-priority work task had no chance to execute yet).
Under such circumstances the user may choose to explicitly create
an ARP entry when receiving a packet using the information from
the packet headers.

  \subsection{Driver Interface}
%TODO
  \subsection{Debugging}
  \lip{} provides limited diagnostics which can prove useful for
  debugging. Since \lip{} is independent from ``normal'' TCP-IP
  none of the standard tools are available (however, sniffing
  communication with \lip{} from another host connected to the
  same LAN using standard tools may produce very valuable information,
  too).

  Diagnostic information consists of several diagnostic entry
  points as well as console messages that may be enabled selectively.

  \subsubsection{Diagnostic Routines}
  The following diagnostic routines are available. They all accept
  a \lipc{FILE} pointer so that the output may be routed to an
  arbitrary file. A \lipc{NULL} value is accepted and results
  in the information being printed on \lipc{stdout}.
  \begin{description}
  \item[\lipc{arpDumpCache()}] Print the contents of the ARP cache
    associated with an interface to a file (or stdout).
  \item[\lipc{lanIpBscDumpConfig()}] Dump the configuration parameters
    of \lip{} but also the amount of corresponding resources that are
    currently in use.
  \item[\lipc{lanIpBscDumpMcGroups()}] Dump information about all
    multicast groups that are subscribed on a given interface.
  \item[\lipc{lanIpBscDumpIfStats()}] Print select information
    associated with an interface. Note that some of this information
    is redundant with \lipc{arpDumpCache()} and \lipc{lanIpBscDumpMcGroups()}.
    An integer ``bitmask'' argument is passed which allows the
    user to select the information she wishes to obtain:
    \begin{description}
      \item[\lipc{IPBSC\_IFSTAT\_INFO\_MAC}]
        Link-level (``ethernet'') statistics.
      \item[\lipc{IPBSC\_IFSTAT\_INFO\_IP}]
        IP related statistics.
      \item[\lipc{IPBSC\_IFSTAT\_INFO\_UDP}]
        UDP related statistics.
      \item[\lipc{IPBSC\_IFSTAT\_INFO\_ARP}]
        ARP related statistics.
      \item[\lipc{IPBSC\_IFSTAT\_INFO\_ICMP}]
        ICMP related statistics.
      \item[\lipc{IPBSC\_IFSTAT\_INFO\_IGMP}]
        IGMP related statistics.
      \item[\lipc{IPBSC\_IFSTAT\_INFO\_MCGRP}]
        IP multicast memberships etc. (like \lipc{lanIpBscDumpMcGroups()}).
      \item[\lipc{IPBSC\_IFSTAT\_INFO\_DRV}]
        Driver-specific statistics (if supported by driver).
      \item[\lipc{IPBSC\_IFSTAT\_INFO\_ALL}]
        All of the above.
    \end{description}
  \item[\lipc{lanIpBscGetStats()}] Obtain summary statistics. Sometimes the user
    wishes to process statistics information further rather than having a printout.
    This entry point provides statistics in binary form.
  \item[\lipc{lanIpBscFreeStats()}] Free resources associated with the object
    returned by \lipc{lanIpBscGetStats()}.
  \end{description}

  \subsubsection{Debug Messages}
  Provided that \lip{} was compiled with the symbol \lipc{DEBUG} defined (default)
  then a global variable \lipc{lanIpDebug} is available (the initial value
  of this variable is assigned the value of \lipc{DEBUG}, usually zero).
  By setting select bits in this variable the generation of messages by the
  protocol handler is enabled:
  \begin{description}
    \item[\lipc{DEBUG\_ARP}] Handling of the ARP related events such as
      arrival of ARP requests or replies, generation or displacement of
      cache entries etc.
    \item[\lipc{DEBUG\_IP}] Handling of IP related events such as the processing
      of received IP headers. 
    \item[\lipc{DEBUG\_ICMP}] Handling of ICMP related events.
    \item[\lipc{DEBUG\_UDP}] Handling of UDP related events such as the
      filtering of arriving datagrams etc.
    \item[\lipc{DEBUG\_TASK}] Events related to driver task(s) such as
      successful startup or shutdown.
    \item[\lipc{DEBUG\_IGMP}] Report IGMP related events.
  \end{description}
  Note that the \lipc{lanIpDebug} feature is considered rather low-level
  ``internal'' business
  and is not exported to the \lipc{<lanIpBasic.h>} header. Consult the
  source code for details.

\section{The udpSock API}
  The {\em udpSock} API provides simplified access to UDP socket
  communication.
  \subsection{Design Goals}
  \begin{itemize}
  \item Provide a simple interface to low-level datagram communication.
  \item Use familiar ``BSD socket''-like abstraction.
  \item Provide some hooks into low-level protocol features (for sake
        of efficiency on low-end CPU).
  \item Give the user access to raw data buffers so that (on select hardware)
        payload data does not have to be copied but can be accessed
        directly by hardware (if it features DMA). 
  \item Support IPV4 multicast groups and IGMP.
  \end{itemize}
  \subsection{Relation to udpComm}
  The {\em udpComm} API (which originally was part of \lip{} but 
  since has been unbundled and moved into the ``EPICS modules''
  realm) is a very slim ``wrapper'' API on top of ``udpSock''.
  Most udpSock API calls translate one-to-one into a corresponding
  udpComm call. However, udpComm is designed to be portable.
  An implementation of udpComm using regular BSD
  sockets is available. This makes it possible to develop applications
  on udpComm and develop/test using the udpCommBSD implementation
  (which exists on every platform with BSD sockets). Only where
  real-time requirements must be met the udpCommBSD layer can
  simply be swapped for \lip{}/udpSock. This architecture has
  greatly helped development, e.g., of the LCLS/BPM implementation
  since most work could be done and tested (non-real time) on
  an ordinary linux workstation (using udpCommBSD).

  \subsection{Programming Interface}
  For a more detailed and up-to-date description of the 
  API please consult the \lipc{lanIpBasic.h} header file.
  Note that some more specialized entry points are not
  listed here (see the aforementioned header file).

  \subsubsection{Socket Creation and Destruction}
  \begin{description}
  \item[\lipc{udpSockCreate()}] Create a communication endpoint
  bound to a specific UDP port.
  \item[\lipc{udpSockDestroy()}] Release resources associated with
  a given socket.
  \end{description}

  \subsubsection{Buffer Management}
  One design goal for \lip{} was to allow the user to interact 
  directly with buffers used for transmission and reception thus
  enabling him/her to avoid unnecessary copying of payload data.
  \begin{description}
  \item[\lipc{udpSockGetBuf()}] Allocate a buffer.
  \item[\lipc{udpSockRefBuf()}] Increment reference count of
  a buffer. The buffer reference-count facilitates sharing
  a single buffer among multiple ``users'' (this is mostly useful
  for shared RX buffers that are treated as ``read-only'').
  \item[\lipc{udpSockFreeBuf()}] Decrement reference count and
  release the buffer when the count drops to zero.
  \item[\lipc{udpSockUdpBufPayload()}] Given a buffer handle
  this routine computes the starting address of the UDP
  payload area inside the (otherwise opaque) buffer.
  \end{description}

  \subsubsection{``Connections''}
  \begin{description}
  \item[\lipc{udpSockConnect()}] UDP is of course not connection
  oriented. The term ``connection'' in this context simply means
  establishing an association with a given peer (IP-address and
  UDP port). The peer is used as a destination when sending
  and received datagrams are accepted only if they are sent
  by the peer (a special mode also accepts datagrams from
  any source if the ``peer''/target of the connection is a
  multicast group).
  \end{description}

  \subsubsection{Data Transmission and Reception}

  \begin{description}
  \item[\lipc{udpSockSendTo()}] Allocates a fresh \rbuf{}, fills
  the \ethn{}, IPV4 and UDP headers and copies the user data to
  the UDP payload area. The destination address (IPV4 address and
  UDP port) is provided by arguments to this call.
  \item[\lipc{udpSockSend()}] Like \lipc{udpSockSendTo()} but the
  socket must be ``connected'', i.e., peer address information has
  already been provided and stored using \lipc{udpSockConnect()}.
  \item[\lipc{udpSockSendBufTo()}] Like \lipc{udpSockSendTo()} but
  the user passes an \rbuf() rather than a pointer to a data area.
  The \rbuf{} must have been allocated by the user previously and
  filled with the UDP payload. Using this entry point avoids
  copying the entire payload into an \rbuf{}.
  \item[\lipc{udpSockSendBuf()}] relates to \lipc{udpSockSendBufTo}
  in the same way as \lipc{udpSockSend} relates to \lipc{udpSockSendTo}.
  \item[\lipc{udpSockRecv()}] Return first datagram that is
  stored in a socket's queue. If none is available then block
  for data to arrive for some user-specified time.
  \item[\lipc{udpSockNRead()}] Return the number of bytes available
  in the socket's receiving queue.
  \end{description}

  \subsubsection{Multicast Groups}
  \begin{description}
  \item[\lipc{udpSockJoinMcast}] Join a multicast group.
  \item[\lipc{udpSockLeaveMcast}] Leave a multicast group.
  \end{description}

\end{document}
